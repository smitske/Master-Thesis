\chapter{Questionnaire}
\label{chapt:appendix}

This questionnaire is part of manageability/usability study of the researched access control models. 
After letting test users complete multiple scenarios with the researched models we ask them to fill out this survey to get an insight in their experiences with the models.

\hskip-2.5cm\section{Part 1: Quantification}
Rate following tasks on a scale of 1 to 5 for how difficult a certain task  was.
1 stands for very easy, 5 for very hard.

\hskip-2.8cm\begin{tabular}{p{8cm} p{1cm} p{1cm} p{1cm} p{1cm} p{1cm}}
    \multicolumn{6}{c}{} \\
    & 1 & 2 & 3 & 4 & 5
    \\
    T1 : Adding new roles to the system.
    & $\square$ & $\square$ & $\square$ & $\square$ & $\square$ \\
    \\
    T2 : Adding new permissions/policies to the system.
    & $\square$ & $\square$ & $\square$ & $\square$ & $\square$ \\
    \\
    T3 : Deciding what a user can do given a set of policies (ABAC).
    & $\square$ & $\square$ & $\square$ & $\square$ & $\square$ \\
    \\
    T4 : Deciding what a user can do given a set of permissions (RBAC) and a mapping of these permissions to roles.
    & $\square$ & $\square$ & $\square$ & $\square$ & $\square$ \\
\end{tabular}
    
\hskip-2.8cm\begin{tabular}{p{8cm} p{1cm} p{1cm} p{1cm} p{1cm} p{1cm}}
    \multicolumn{6}{c}{} \\
    & 1 & 2 & 3 & 4 & 5 \\
    \\
    T5 : Adding new policies and knowing their effects(ABAC).
    & $\square$ & $\square$ & $\square$ & $\square$ & $\square$ \\
    \\
    T6 : Using conditions to prevent access to certain subsets of objects.
    & $\square$ & $\square$ & $\square$ & $\square$ & $\square$ \\
    \\
    T7 : Preventing certain users from seeing specific values.
     & $\square$ & $\square$ & $\square$ & $\square$ & $\square$
    \\
    T8 : Switching your own role to get a different set of permissions.
     & $\square$ & $\square$ & $\square$ & $\square$ & $\square$ \\
    \\
    T9 : Setting up the system with as few permissions as possible(RBAC).
     & $\square$ & $\square$ & $\square$ & $\square$ & $\square$ \\
    \\
     T10 : Setting up the system with as few policies as possible(ABAC).
     & $\square$ & $\square$ & $\square$ & $\square$ & $\square$ \\
    \\
    T11 : Adding roles to users.
     & $\square$ & $\square$ & $\square$ & $\square$ & $\square$ \\
    \\
    T12 : Adding multiple permissions/policies of the same action on different resources.
     & $\square$ & $\square$ & $\square$ & $\square$ & $\square$ \\
     \\
    T13 : Adding permissions/policies with multiple actions to the same resource.
     & $\square$ & $\square$ & $\square$ & $\square$ & $\square$ \\   
     \\
    T14 : Adding multiple masks on a permission/policy on a single resource.
     & $\square$ & $\square$ & $\square$ & $\square$ & $\square$ \\
     \\
    T15 : Adding multiple conditions on a permission/policy on a single resource.
     & $\square$ & $\square$ & $\square$ & $\square$ & $\square$ \\
     \\
    T16 : Deleting any of the created roles, policies or permissions.
     & $\square$ & $\square$ & $\square$ & $\square$ & $\square$ \\
     \\
    T17 : Using negative policies to limit the number of total policies.
     & $\square$ & $\square$ & $\square$ & $\square$ & $\square$ \\
     \\
    T18 : 
     & $\square$ & $\square$ & $\square$ & $\square$ & $\square$ \\
     \\
    T19 : 
     & $\square$ & $\square$ & $\square$ & $\square$ & $\square$ \\
     \\
    T20 : 
     & $\square$ & $\square$ & $\square$ & $\square$ & $\square$ \\
\end{tabular}

\clearpage

\hskip-2.5cm\section{Part 2: Statements}
Rate the following statements on how much you agree/disagree with them. 
For these questions we have also looked at similar work on comparison of access control models such as Comparative analysis of Role Base and Attribute Base
Access Control Model in Semantic Web from Sonu Verma et al, they compared the same models but did so in Semantic Web\cite{Related1}. We also looked at Shabnam Mohammad Hasani et al their work, Criteria Specifications for the Comparison and Evaluation of Access Control Models, comparing RBAC with DAC and MAC\cite{Related3}.
 
\hskip-2.8cm\begin{tabular}{p{5cm} c c c c c}
    \multicolumn{6}{c}{} \\        
    & Strongly agree & Agree & Neutral  & Disagree & Strongly Disagree \\ 
    \\
    Q1 : When building heavy role dependant systems the RBAC model is easier to manage than the ABAC model. & $\square$ & $\square$ & $\square$ & $\square$ & $\square$ \\
    \\
    Q2 : Looking at the architecture it would be easy to make a good visual representation of what a user can do using RBAC.
    & $\square$ & $\square$ & $\square$ & $\square$ & $\square$\\
    \\
    Q3 : Looking at the architecture it would be easy to figure out for users what they can do using ABAC.
    & $\square$ & $\square$ & $\square$ & $\square$ & $\square$\\
    \\
    Q4 : The addition of negative rules in ABAC makes it easier to manage because we can make global policies and exempt specific cases.
    & $\square$ & $\square$ & $\square$ & $\square$ & $\square$\\
\end{tabular}

\hskip-2.8cm\begin{tabular}{p{5cm} c c c c c}
    \multicolumn{6}{c}{} \\        
    & Strongly agree & Agree & Neutral  & Disagree & Strongly Disagree \\
    \\
    Q5 : Adding permissions and assigning these to roles in RBAC has little to no unforeseen side effects.
    & $\square$ & $\square$ & $\square$ & $\square$ & $\square$\\
    \\
    Q6 : Adding new policies in ABAC has little to no unforeseen side effects.
    & $\square$ & $\square$ & $\square$ & $\square$ & $\square$\\
    \\
    Q7 : The implemented RBAC model allows for access control that is dynamic enough to depend on it with a running system.
    & $\square$ & $\square$ & $\square$ & $\square$ & $\square$\\
    \\
    Q8 : The implemented ABAC model allows for access control that is dynamic enough to depend on it with a running system.
    & $\square$ & $\square$ & $\square$ & $\square$ & $\square$\\
    \\
    Q9 : The addition of constraints, where users cannot have 2 specific roles assigned to them, in the RBAC model are worthwhile.
    & $\square$ & $\square$ & $\square$ & $\square$ & $\square$\\
    Q10 : There is a problem with policy explosion for the ABAC model.
    & $\square$ & $\square$ & $\square$ & $\square$ & $\square$\\
    \\
    Q10 : There is a problem with role explosion for the RBAC model.
    & $\square$ & $\square$ & $\square$ & $\square$ & $\square$\\
    \\
    Q11 : There is a problem with permission explosion for the RBAC model.
    & $\square$ & $\square$ & $\square$ & $\square$ & $\square$\\
\end{tabular}

\hskip-2.8cm\begin{tabular}{p{5cm} c c c c c}
    \multicolumn{6}{c}{} \\        
    & Strongly agree & Agree & Neutral  & Disagree & Strongly Disagree \\
    \\
    Q12 : The ability to change roles quickly  allows for better separation of duty.
    & $\square$ & $\square$ & $\square$ & $\square$ & $\square$\\
    \\
    Q13 : It is better to put access control authority in the hands of an administrator than in the hands of the users themselves like is done in other models.
    & $\square$ & $\square$ & $\square$ & $\square$ & $\square$\\
    \\
    Q14 : Having a role hierarchy where we can inherit from other roles would make management easier.
    & $\square$ & $\square$ & $\square$ & $\square$ & $\square$\\
    \\
    Q15 : The RBAC model allows for easy customization for different applications.
    & $\square$ & $\square$ & $\square$ & $\square$ & $\square$\\
    \\
    Q16 : The ABAC model allows for easy customization for different applications.
    & $\square$ & $\square$ & $\square$ & $\square$ & $\square$\\
    \\
    Q17 : The RBAC model is better suited for complex systems than the ABAC model.
    & $\square$ & $\square$ & $\square$ & $\square$ & $\square$\\
    \\
    Q18 : Maintaining (adding rules in an existing system) a RBAC system is less complicated than maintaining an ABAC system.
    & $\square$ & $\square$ & $\square$ & $\square$ & $\square$\\
    \\
    Q19 : The options for setting conditions on policies and permissions are versatile enough.
    & $\square$ & $\square$ & $\square$ & $\square$ & $\square$\\
    \\
    Q20 : Using ABAC the behaviour is not always clear from just looking at the different policies.
    & $\square$ & $\square$ & $\square$ & $\square$ & $\square$\\
\end{tabular}

\hskip-2.8cm\begin{tabular}{p{5cm} c c c c c}
    \multicolumn{6}{c}{} \\        
    & Strongly agree & Agree & Neutral  & Disagree & Strongly Disagree \\
    \\
    Q21 : Using conditions requires significant knowledge of the inner workings of the application.
    & $\square$ & $\square$ & $\square$ & $\square$ & $\square$\\
    \\
    Q22 : Masking specific values requires significant knowledge about the data model used for the application.
    & $\square$ & $\square$ & $\square$ & $\square$ & $\square$\\
    \\
    Q23 : Using features such as masking and conditions require a deep knowledge of the implementation of the access control model.
    & $\square$ & $\square$ & $\square$ & $\square$ & $\square$\\
    \\
    Q24 : To allow easy editing of conditions there is need of a domain specific editing language.
    & $\square$ & $\square$ & $\square$ & $\square$ & $\square$\\
    \\
    Q25 : Putting the responsibility for maintaining the access control models in the hands of an admin can be more error prone than putting it in the hands of the developers.
    & $\square$ & $\square$ & $\square$ & $\square$ & $\square$\\
    \\
    Q26 : The flexibility to change access control dynamically compared to programming it leads to less development time.
    & $\square$ & $\square$ & $\square$ & $\square$ & $\square$\\
    \\
    Q27 : The flexibility to change access control dynamically compared to programming it leads to more setup time.
    & $\square$ & $\square$ & $\square$ & $\square$ & $\square$\\

\end{tabular}



