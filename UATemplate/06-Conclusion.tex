\chapter{Conclusions}
\label{chapt:Conclusions}
Out of the experiment we have done multiple conclusions can be made, some in preference of the attribute based model and others in preference of the attribute enhanced role based model.
First of all we can conclude that the role based access control model does not suffice if we need a more fine grained solution to access control.
This can however be resolved by using combined models such as the attribute enhanced role based access control model.
We can also conclude that the attribute based model is more user friendly when using a very basic user interface, if we spend more time on the interface however the users of the system see the attribute enhanced role based model to benefit most and become easier to use.
When we compared the two models with each other we came to the conclusion that the attribute based model allowed users the set up the system faster than the attribute enhanced role based model.
The flip side of this is that the attribute based model is more error sensitive, where a mistake in any rule can cause problems for every access attempt of those objects, this is not the case for the attribute enhanced role based model.
On the level of maintainability the users concluded that when the requirements changed the attribute enhanced role based model is more robust to these changes and easiest to adapt to these changes.
The users found the attribute enhanced role based model to be the most straightforward model to use as a developer to make it clear for the other users of the system what their rights are within the system.
Performance wise we conclude that much of the performance depends on the rules that have to be evaluated by the system with each access attempt.
Generally this number is lower when using the attribute enhanced role based model with permissions while the attribute based model with policies requires more policies to be checked, leading to poorer performance.

\section{Related work}
There are a wide variety of studies that have been conducted to compare different access control models in different circumstances and environments.
Sonu Verma et al have done a comparison of role based access control and attribute based access control in semantic web in Comparative analysis of Role Base and Attribute Base Access Control Model in Semantic Web \cite{Related1}.
This study showed how role based access control itself is not as fine grained as attribute based access control which is why we also took a different model.
Shabnam Mohammad Hasani et al have done an extensive study in Criteria Specifications for the Comparison and
Evaluation of Access Control Models on how we can compare the different access control models \cite{Related3} from which we also used criteria in our study.
In Mohammed ENNAHBAOUI et al. their work Study of Access Control Models \cite{Related2} a comparison between the higher level models is done, mandatory access control, discretionary access control, role based access control and organizational based access control.



\section{Future work}
A lot of future work can still be done on the subject, we have only done a comparison between two models but we can compare these with the other models mentioned in chapter two.
An expansion can be made that compares the chosen models with the task role based access control since this model also meets the requirements set up by the tested system.
Other future work lies in creating more new models by combination of the features of both role based access control and attribute based access control.
We used one such combination in this study but many more can be made, as explained by Timothy R Weil et al in Adding attributes to role based access control \cite{Combined1}.
They should first be worked out theoretically since not much has been done on that subject, we mentioned two of the existing models in chapter two.
This can be in conjunction with a case study.